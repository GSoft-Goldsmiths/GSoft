\documentclass[12pt,a4paper]{article}
  \usepackage[toc,page]{appendix}
  \usepackage{longtable}
  \usepackage{listings} 
  \usepackage{verbatim}
  \usepackage{graphicx}
  \usepackage{tabularx}
  \usepackage{subfig}
  \usepackage{float}
  \usepackage{csvsimple}
  \begin{document}
    \begin{titlepage}
      \centering
      {\scshape\LARGE Goldsmiths, University of London \par}
      \vspace{1cm}
      {\scshape\Large Software project final report\par}
      \vspace{1.5cm}
      {\huge\bfseries iLost\par}
      \vspace{2cm}
      {\Large\itshape 
        Ahmed, Muhammad\\
        Chowdhury, Thairan\\
        Davies Minta, Dylan\\     
        Fakrul, Mahmudul\\    
        Farkhani, Hussein\\ 
        Jheng-Hao, Lin\\
        Pecorella, Mariano\\ \par}
      \vfill
      supervised by\par
      \textsc{Tim Blackwell} 
      \vfill
      % Bottom of the page
      {\large \today \par}
    \end{titlepage}

    \tableofcontents
    \newpage

    \section{Introduction}
    \section{Development Record}
    \newpage
    \section{Formative Evaluation}
      % total 1650 words
      \subsection{iOS App Evaluation} 
          % total 1000 words
        \subsubsection{Objectives and Questions}
          % 100 words
          \paragraph{}
            In order to test our app usability, we chose some of the user storise which the functionalities were already completed at that moment and .

        \subsubsection{Participants, Location and Setup}
          % 100 words
          \paragraph{}
          According to Jakob Nielsen, testing 5 users in a usability study could find almost as many usability problems as testing more participants\cite{LostAndFound} . So iLost app was tested with 5 participants for each version. The study was taken place in the library and refactory of the Goldsmiths,University of London, and the participants were the students who used to bring a bag to the campus daily. Participants were provided with a iPhone 6 to try out the app and an iPad to fill the online questionnair after using the app. 
          
        \subsubsection{Methodology and Measures}
          % 100 words
          \paragraph{}
            There was an observesr to guide the user through the test and take notes of how the user used iLost app, specially when the user was confused or couldn't get the task done.
        \subsubsection{iOS app v0.10 Evaluation}
          150 max words
        \subsubsection{iOS app v0.11 Evaluation}
          150 max words
        \subsubsection{iOS app v0.12 Evaluation}
          150 max words
      \subsection{Tracker Evaluation}
          total 700 words
        \subsubsection{Objectives and Questions}
          100 words
        \subsubsection{Location, Setup and Participants}
          100 words
        \subsubsection{Methodology and Measures}
          100 words
        \subsubsection{Tracker v0.10 Evaluation}
          150 max words
        \subsubsection{Tracker v0.11 Evaluation}
          150 max words
      \subsection{Conclusion}
        200 max words
      \newpage
    \section{Design and Implementation}
    \section{QUality Assurance}
    \section{Summative Evaluation}
    
    \begin{thebibliography}{20}
      % \addcontentsline{toc}{chapter}{Bibliography}
      \bibitem{LostAndFound} "How Many Test Users in a Usability Study?", Nielsen Norman Group, 2012. [Online]. Available: https://www.nngroup.com/articles/how-many-test-users/. [Accessed: 01- Mar- 2018].
    \end{thebibliography}
    
    \begin{appendices}
      \section{Tasks Divided}
        \paragraph{}
      \section{Progress Tracking Form}

    \end{appendices}

  \end{document}


